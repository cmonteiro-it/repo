%%%%%%%%%%%%%%%%%%%%%%%%%%%%%%%%%%%%%%%%%
% Engineering Calculation Paper
% LaTeX Template
% Version 1.0 (20/1/13)
%
% This template has been downloaded from:
% http://www.LaTeXTemplates.com
%
% Original author:
% Dmitry Volynkin (dim_voly@yahoo.com.au)
%
% License:
% CC BY-NC-SA 3.0 (http://creativecommons.org/licenses/by-nc-sa/3.0/)
%
%%%%%%%%%%%%%%%%%%%%%%%%%%%%%%%%%%%%%%%%%

%----------------------------------------------------------------------------------------
%	PACKAGES AND OTHER DOCUMENT CONFIGURATIONS
%----------------------------------------------------------------------------------------

\documentclass[9pt,a4paper]{article} % Use A4 paper with a 12pt font size - different paper sizes will require manual recalculation of page margins and border positions

%\usepackage{marginnote} % Required for margin notes
\usepackage{wallpaper} % Required to set each page to have a background
\usepackage{lastpage} % Required to print the total number of pages
\usepackage[left=1.3cm,right=4.8cm,top=1.8cm,bottom=4.0cm,marginparwidth=5cm]{geometry} % Adjust page margins
\usepackage{amsmath} % Required for equation customization
\usepackage{amssymb} % Required to include mathematical symbols
\usepackage{xcolor} % Required to specify colors by name

\usepackage{fancyhdr} % Required to customize headers
\setlength{\headheight}{80pt} % Increase the size of the header to accommodate meta-information
\pagestyle{fancy}\fancyhf{} % Use the custom header specified below
\renewcommand{\headrulewidth}{0pt} % Remove the default horizontal rule under the header

\setlength{\parindent}{0cm} % Remove paragraph indentation
\newcommand{\tab}{\hspace*{2em}} % Defines a new command for some horizontal space

\newcommand\BackgroundStructure{ % Command to specify the background of each page
\setlength{\unitlength}{1mm} % Set the unit length to millimeters

\setlength\fboxsep{0mm} % Adjusts the distance between the frameboxes and the borderlines
\setlength\fboxrule{0.5mm} % Increase the thickness of the border line
\put(10, 10){\fcolorbox{black}{white!30}{\framebox(155,247){}}} % Main content box
\put(165, 10){\fcolorbox{black}{blue!10}{\framebox(37,247){}}} % Margin box
\put(10, 259){\fcolorbox{black}{white!10}{\framebox(192,35){}}} % Header box
%\put(137, 263){\includegraphics[height=23mm,keepaspectratio]{logo}} % Logo box - maximum height/width:
}

%----------------------------------------------------------------------------------------
%	HEADER INFORMATION
%----------------------------------------------------------------------------------------

\fancyhead[L]{\begin{tabular}{l r | l r} % The header is a table with 4 columns
\textbf{Lehrgangsnummer} & 0188.02 & \textbf{Seite} & \thepage \\ % Project name and page count
\textbf{Vertragsnummer} & 800438834 & \textbf{Korrekturdatum} &  \\ % Job number and last updated date
\textbf{Name, Vorname} & Morais Bennemann, Cristiano & \textbf{Punkte} & /100 \\ % Job number and last updated date
\textbf{Code} & CCOM04B-XX1-K02 & \textbf{Note} &  				\\ % Version and reviewed date
\textbf{Postleitzahl und Ort} & 79108 - Freiburg & \textbf{Lehrer} & Thorsten Schreiber \\ % Designer and reviewer
	\textbf{Straße} & Zähringer Str. 329A & \textbf{Unterschrifft Lehrer} & \hrulefill \hrulefill \hrulefill
 \\ % Designer and reviewer
\end{tabular}}

%----------------------------------------------------------------------------------------

\begin{document}

\AddToShipoutPicture{\BackgroundStructure} % Set the background of each page to that specified above in the header information section

%----------------------------------------------------------------------------------------
%	DOCUMENT CONTENT
%----------------------------------------------------------------------------------------
\marginpar{Punktvergabe:}
\large \texttt{Datum: \today }
\section*{Aufgabe 1}
Teil des Lebenszyklus von Software sind die:
 \begin{itemize}
	 \item Konfiguration von Software
	 \item Bereitstellung Von Software
	 \item Pflege von Software
	 \item Ablösung von Software
 \end{itemize}
Dieser Softwarelebenzyklus führt bei einer wachsenden Anzahl von Anwendungen und bei dem Einsatz von verschiedenen Versionen der selben Software zu einem stetig steigenden Arbeitsaufwand und hohen Betriebskosten. Um die Unternehmen bzw. IT-Abteilung zu entlassten und Kosten zu minimieren hat sich das Application-Service-Provider Modell etabliert das als vorlaufmodell für die heutigen SaaS-Anbieter gilt.

\section*{Aufgabe 2}
Ein ASP (Application-Service-Provider) bietet eine Betreuung rund um den Softwarelebenszyklus. Dies bedeutet, dass anstatt der internen IT-Abteilung eines Unternehmens ein Dienstleister verantwortlich für die Pflege, Konfiguration, Bereitstellung und Ablösung von Software ist. Damit wird der ganze Lebenszyklus-Prozess ``outgesourct''  und Betiebskosten gesenkt. Zusätzlich können ASPs auch die Aktualisierung von Software und die Datensicherung übernehmen. ASPs stellen auch meistens eine Hotline für die Kunden zur Verfügung.

\section*{Aufgabe 3}
Vier Wirtschaftliche Vorteile von SaaS-Angeboten gegenüber ASP-Angeboten sind:
\begin{enumerate}
	\item Da der Softwarehersteller seine Software auf seiner eigenen Plattform bereitstellt, wird die Software gezielt und spezifisch für diese Plattform entwickelt. Wodurch die Anpassung, Entwicklung und das Testen für verschiedene Plattformen wie z.B verschiedene Datenbanksysteme und Betriebssysteme wegfällt
	\item Da der Softwarehersteller seine Software selbst anbietet ist klar das er Verwendungszweck und die Bereitstellungsform gezielt während der Entwicklung der Software berücksichtigt. Deswegen werden z.B sämtliche Anpassungen bezüglich der Skalierbarkeit der Anwendung schon während dem Entwicklungsprozess durchgeführt.
	\item Da die Kosteneinsparung Ziel von jedem Unternehmen ist, wird der Softwarehersteller seine Plattform so auslegen, sodass so viele Kunden wie möglich bedient werden können. Dies gleichzeitig mit der möchst höchsten Qualität ohne zusätzliche Hardware. Dieser Zwang treibt den Softwarehersteller eine Mandantenfähigkeit in sein SaaS-Produkt einzubauen. Diese Mandanten fähig muss dann nicht mehr durch Beispielweise, ASPs hinzugefügt werden. \newline

		\vspace{3mm}

	\item SaaS-Angebote weden häufig über den Browser bereitgestellt. Da so gut wie jeder einen Browser verfügt, und die Hardware Voraussetzungen für einen aktuellen Browser gering sind fallen sämtliche zusätzliche Kosten die für alternative Bereitstellungstechnolgien wie z.B der Zugriff über einen Terminal-Server aus. Dies führt zu weniger Kosten.
\end{enumerate}

\section*{Aufgabe 4}
Web-Hosting Unternehmen haben bereits die notwendige Erfahrung für die Erstellung und den Betrieb eines PaaS-Angebotes. IaaS-Angebote werden oft durch frühere ASPs angeboten da diese viel Erfahrung mit der Bereitstellung und Betreuung von Infrastrukuren besitzen. SaaS-Angebote werden direkt vom Softwarehersteller angeboten weil diese am besten spezilisiert und all das notwendige Wissen über ihre Software besitzen und deswegen am besten für die Bereitstellung geeignet sind.

\section*{Aufgabe 5}
Es handelt sich um ein Off-premise Public Cloud, SaaS-Angebot.

\section*{Aufgabe 6}
Weil diese die Möglichkeit bieten die Kundendaten vor dem Upload zu verschlüsseln im Gegensatz z.B Dropbox wo nur der Transport der Daten verschlüsselt wird und ``Data-at-Rest'' unverschlüsselt bleibt im remote Rechenzentrum.

\section*{Aufgabe 7}
Die Lokale Installation von Excel stellt erweiterte Importfunktionen der Daten zur Verfügung. Zusätzlich lassen sich nn der lokalen Version von Word Inhaltsverzeichnisse erstellen. Diese beiden funktionen sind in Office 365 nicht vorhanden.





%----------------------------------------------------------------------------------------

\end{document}
